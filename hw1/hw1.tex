\documentclass[12pt, letterpaper]{article}
\usepackage[top = 1cm, bottom = 0.75cm, left = 1in, right = 1in]{geometry}
\usepackage[utf8]{inputenc}
\usepackage[vietnamese]{babel}
\usepackage{graphicx}
\usepackage{amsmath}
\usepackage{titlesec}
\usepackage{color}

\titleformat{\section}
{\LARGE\bfseries}
{Bài \thesection: }
{0em}
{}

\titleformat{\subsection}
{\Large\bfseries}
{{\thesubsection} }
{0em}
{}

\renewcommand{\theenumi}{\bfseries\large\alph{enumi}}

\title{
  \large\textbf{TRƯỜNG ĐẠI HỌC CÔNG NGHỆ THÔNG TIN} \\
  \large\textbf{KHOA KHOA HỌC MÁY TÍNH} \\
  \vfill
  \begin{figure}[h]
    \centering 
    \includegraphics[width=0.5\linewidth]{uit} 
  \end{figure}
  \vfill
  \textbf{BÀI TẬP TUẦN 1} \\
  \textbf{MÔN\@: PHÂN TÍCH VÀ THIẾT KẾ THUẬT TOÁN}\\
  \vspace{1cm}
  \large \textcolor{blue}{\textbf{ĐÁNH GIÁ THUẬT TOÁN DÙNG KỸ THUẬT TOÁN SƠ CẤP \\}}
  \vfill
}

\author{
  \begin{tabular}{rl}
    \textbf{GV hướng dẫn:} &Huỳnh Thị Thanh Thương \\
    \textbf{Nhóm thực hiện:}
    &1. Võ Đình Khánh 22520659 \\
    &2. Nguyễn Gia Bảo 22520108 \\
    &3. Nguyễn Trần Phúc 22521135 \\
    &4. Hồ Trọng Duy Quang 22521200 \\
  \end{tabular}
}

\date{{\vfill}Thành phố Hồ Chí Minh, \MakeLowercase{\today}}

\begin{document}
  \maketitle
  \pagebreak
  \newgeometry{margin=3cm}

  \section{Tính tổng hữu hạn}

  \subsection{Yêu cầu: Tính chính xác, không cho phép sai số hay xấp xỉ}
  \begin{enumerate}
      \item $ \begin{aligned}[t]
        1 + 3 + 5 + 7 + \cdots + 999 
        & = \sum^{499}_{i = 0} (2i + 1) \\
        & = 2 \sum^{499}_{i = 0} i + \sum^{499}_{i = 0} 1 \\
        & = 2 \frac{499(499 + 1)}{2} + (499 - 0 + 1) \\
        & = 250000 \\
      \end{aligned} $

      \item $ \begin{aligned}[t]
        2 + 4 + 8 + 16 + \cdots + 1024
        & = \sum^{10}_{i = 1} 2^i \\
        & = \sum^{10}_{i = 0} 2^i - 2^0 \\
        &= 2^{10 + 1} - 1 - 1 \\
        &= 2046 \\
      \end{aligned} $

      \item $ \begin{aligned} 
        \sum^{n + 1}_{i = 3} 1 = (n + 1) - 3 + 1 = n - 1
      \end{aligned} $

      \item $ \begin{aligned}[t]
        \sum^{n + 1}_{i = 3} i 
        & = \sum^{n + 1}_{i = 1} i - \sum^{2}_{i = 1} i \\
        & = \frac{(n + 1)(n + 2)}{2} - \frac{2(2 + 1)}{2} \\
        & = \frac{n^2 + 3n - 4}{2} \\
      \end{aligned} $

      \item $ \begin{aligned}[t]
        \sum^{n - 1}_{i = 0} i(i + 1)
        & = \sum^{n-1}_{i = 0} i^2 + \sum^{n - 1}_{i = 0} i \\
        & = \sum^{n-1}_{i = 1} i^2 + \sum^{n - 1}_{i = 1} i \\
        & = \frac{(n - 1)n(2n - 1)}{6} + \frac{(n - 1)n}{2} \\
        & = \frac{n(n - 1)(n + 1)}{3} \\
      \end{aligned} $


    \item $ \begin{aligned}[t]
      \sum^{n}_{j = 1} 3^{j + 1}
      & = 3\sum^{n}_{j = 1} 3^j \\
      & = 3[(\sum^{n}_{j = 0} 3^j) - 3^0] \\
      & = 3(\sum^{n}_{j = 0} 3^j) - 3 \\
      & = 3\frac{3^{n + 1} - 1}{3 - 1} - 3 \\
      & = \frac{3^{n + 2} - 9}{2} \\
    \end{aligned} $

    \item $ \begin{aligned}[t]
      \sum^{n}_{i = 1} \sum^{n}_{j = 1} ij
      & = \sum^{n}_{i = 1} i [\frac{n(n + 1)}{2}] \\ 
      & = \frac{n(n + 1)}{2} \cdot \frac{n(n + 1)}{2} \\
      & = \frac{n^2{(n + 1)}^2}{4}
    \end{aligned} $

    \item $ \begin{aligned}[t]
      \sum^{n}_{i = 1} \frac{1}{i(i + 1)}
      & = \sum^{n}_{i = 1} \frac{1}{i} - \sum^{n}_{i = 1} \frac{1}{i + 1} \\
      & = \sum^{n - 1}_{i = 0} \frac{1}{i + 1} - \sum^{n}_{i = 1} \frac{1}{i + 1} \\
      & = 1 - \frac{1}{n + 1}
    \end{aligned} $

    \item $ \begin{aligned}[t]
      \sum_{j \in \{2, 3, 5\}} (j^2 + j) = (2^2 + 2) + (3^2 + 3) + (5^2 + 5) = 48
    \end{aligned} $

    \item $ \begin{aligned}[t]
        \sum^{m}_{i = 1} \sum^{n}_{j = 0} \sum^{100}_{k = 0} (i + j)
        & = 101 \sum^{m}_{i = 1} \sum^{n}_{j = 0} (i + j) \\
        & = 101 \sum^{m}_{i = 1} [i(n + 1) + \frac{n(n + 1)}{2}] \\
        & = \frac{101m(n + 1)(m + 1)}{2} + \frac{101mn(n + 1)}{2} \\
        & = \frac{101m(n + 1)(m + n + 1)}{2}
    \end{aligned} $

  \end{enumerate}

  \subsection{Yêu cầu: Tính chính xác được thì tốt, không thì cho phép tính gần đúng/xấp xỉ}
  \begin{enumerate}
    \item $ \sum^{n - 1}_{i = 0} {(i^2 + 1)}^2 $
    \item $ \sum^{n - 1}_{i = 2} \lg i^2 $
    \item $ \sum^{n}_{i = 1} (i + 1)2^{i - 1} $
    \item $ \sum^{n - 1}_{i = 0} \sum^{i - 1}_{j = 0} (i + j) $
  \end{enumerate}

  \section{Đếm số phép gán và so sánh}
  \section{Đếm số phép gán và so sánh}

\end{document}
