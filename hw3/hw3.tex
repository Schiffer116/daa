\documentclass[12pt, letterpaper]{article}
\usepackage[top = 1cm, bottom = 0.75cm, left = 1in, right = 1in]{geometry}
\usepackage[utf8]{inputenc}
\usepackage[vietnamese]{babel}
\usepackage{graphicx}
\usepackage{amsmath}
\usepackage{titlesec}
\usepackage{listings}
\usepackage{adjustbox}
\usepackage{enumitem}
\usepackage{hyperref}
\usepackage{amssymb}
\usepackage{array}
\usepackage{tikz}
\geometry{margin=3cm}

\titleformat{\section}
{\Large\bfseries}
{Bài tập \thesection: }
{0em}
{}

\titleformat{\subsection}
{\large\bfseries}
{{\thesubsection} }
{0em}
{}

\lstset{
  basicstyle=\ttfamily,
  mathescape
}

\title{
  \large\textbf{TRƯỜNG ĐẠI HỌC CÔNG NGHỆ THÔNG TIN} \\
  \large\textbf{KHOA KHOA HỌC MÁY TÍNH} \\
  \vfill
  \begin{figure}[h]
    \centering
    \includegraphics[width=0.5\linewidth]{uit}
  \end{figure}
  \vfill
  \textbf{BÀI TẬP MÔN PHÂN TÍCH VÀ THIẾT KẾ THUẬT TOÁN} \\
  \vspace{1cm}
  \Large \textbf{HOMEWORK \#03:} \\
  \large \text{ĐỘ PHỨC TẠP VÀ CÁC KÝ HIỆU TIỆM CẬN}
  \vfill
}

\author{
  \begin{tabular}{rl}
    \textbf{GV hướng dẫn:} &Huỳnh Thị Thanh Thương \\
    \textbf{Nhóm thực hiện:}
    &1. Nguyễn Gia Bảo 22520108 \\
    &2. Võ Đình Khánh 22520659 \\
    &3. Nguyễn Trần Phúc 22521135 \\
    &4. Hồ Trọng Duy Quang 22521200 \\
  \end{tabular}
}

\date{{\vfill}Thành phố Hồ Chí Minh, \MakeLowercase{\today}}

\begin{document}
\maketitle
\pagebreak
\newgeometry{margin=3cm}
\section{Bài tập 1: So sánh thời gian chạy}
\renewcommand{\arraystretch}{1.25}
\begin{table}[h]
    \centering
    \resizebox{\textwidth}{!}{
    \begin{tabular}{llllllll}
    \hline
    \textbf{} & \textbf{1 second} & \textbf{1 minute} & \textbf{1 hour} & \textbf{1 day} & \textbf{1 month} & \textbf{1 year} & \textbf{1 century} \\
    \hline
    $\lg n$ & $2^{10^6}$ & $2^{6\cdot10^7}$ & $2^{36\cdot10^8}$ & $2^{864\cdot10^8}$ & $2^{25920\cdot10^8}$ & $2^{315360\cdot10^8}$ & $2^{31556736\cdot10^8}$ \\
    $\sqrt{n}$ & $10^{12}$ & $36 \cdot 10^{14}$ & $1296 \cdot 10^{16}$ & $746496 \cdot 10^{16}$ & $6718464 \cdot 10^{18}$ & $994519296 \cdot 10^{18}$ & $995827586973696 \cdot 10^{16}$ \\
    $n$ & $10^6$ & $6 \cdot 10^7$ & $36 \cdot 10^8$ & $864 \cdot 10^8$ & $2592 \cdot 10^9$ & $31536 \cdot 10^9$ & $31556736 \cdot 10^8$ \\
    $n \lg n$ & 62746 & 2801417 & 133378058 & 2755147513 & 71870856404 & 797633893349 & 68654697441062 \\
    $n^2$ & 1000 & 7745 & 60000 & 293938 & 1609968 & 5615692 & 56175382 \\
    $n^3$ & 100 & 391 & 1532 & 4420 & 13736 & 31593 & 146677 \\
    $2^n$ & 19 & 25 & 31 & 36 & 41 & 44 & 51 \\
    $n!$ & 9 & 11 & 12 & 13 & 15 & 16 & 17 \\
    \hline
    \end{tabular}}
    \end{table}
    \begin{minipage}{1\textwidth}
    Gọi \( T(n) \) là thời gian thực hiện (micro giây) \( n \) bài toán. Từ đó, ta có:
\[
    \begin{aligned}
        T_1 &= n \log n \\
        T_3 &= n^3 \\
        T_4 &= 2^n \\
        T_5 &= n!
    \end{aligned}
\]

Gọi \( N \) là tối đa số bài toán các hàm \( T(n) \) có thể thực hiện trong 1 giây ( \( 10^6 \) micro giây). Từ đó, ta có:
\[
\Rightarrow
\left\{
\begin{aligned}
T_1 &= n \log n = 10^6 \\
T_2 &= n^3 = 10^6 \\
T_3 &= 2^n = 10^6 \\
T_4 &= n! = 10^6
\end{aligned}
\right.
\]\\
\[
\Rightarrow
\left\{
\begin{aligned}
        N &= 2^{10^6} \\
        N &= (10^6)^{\frac{1}{3}} = 10^2 \\
        N &= \log(10^6) = 6 \log(10) = 19 \\
        N &= 9\\
\end{aligned}
\right.
\]
\end{minipage}
Tương tự với T = 1 năm = 31557600 giây, T = 1 thế kỉ = 3155760000 ta có:
\[
\Rightarrow
\textbf{T là 1 năm:}
\left\{
\begin{aligned}
        N &= 2^{10^6}\cdot31557600\\
        N &=  10^2\cdot31557600 = 3155760000 \\
        N &= 19\cdot31557600 = 599594400\\
        N &= 9\cdot31557600 = 284018400\\
\end{aligned}\\
\right.\\
\]
\[
\textbf{T là 1 thế kỉ:}
\left\{
\begin{aligned}
        N &= 2^{10^6}\cdot3155760000\\
        N &=  10^2\cdot3155760000 = 315576000000 \\
        N &= 19\cdot3155760000 = 59959440000\\
        N &= 9\cdot3155760000 = 28401840000\\
\end{aligned}
\right.
\]
\section{Với mỗi nhóm bên dưới, hãy sắp xếp tăng dần "theo Big-O nhỏ nhất", có giải thích ngắn gọn cách so sánh}
\textit{Group1}\\
$\begin{aligned}
    f_1(n) &= \binom{n}{100} = \dfrac{n!}{100!(n-100)!} = \dfrac{n(n-1)(n-2)...(n-99)}{100!} = O(n^{100}) \\
    f_2(n) &= n^{100} = O(n^{100}) \\
    f_3(n) &= \dfrac{1}{n} = n^{-1} = O(n^{-1}) \\
    f_4(n) &= 10^{1000} n = O(n) \\
    f_5(n) &= nlogn = O(n^{1 + c}) \quad (\text{vì } logn = O(n^c), \text{c có thể rất nhỏ}) \\
\end{aligned}$ \\ \\
$\Rightarrow f_3(n) < f_4(n) < f_5(n) < f_2(n) = f_1(n) \quad (\text{vì } O(n^{-1}) < O(n) < O(n^{1+c}) < O(n^{100}))$\\ \\
\textit{Group2}\\
$\begin{aligned}
    f_1(n) &= 2^{2^{1000000}} = O(1) \\
    f_2(n) &= 2^{100000n} = O((2^{100000})^n) \\
    f_3(n) &= \binom{n}{2} = \dfrac{n!}{2!(n-2)!} = \dfrac{n(n-1)}{2} = O(n^2) \\
    f_4(n) &= n\sqrt{n} = n^{\frac{3}{2}} = O(n^{\frac{3}{2}})\\
\end{aligned}$ \\ \\
$\Rightarrow f_1(n) < f_4(n) < f_3(n) < f_2(n) \quad (\text{vì } O(1) < O(n^{\frac{3}{2}}) < O(n^2) < O(a^n) \text{ với a > 1})$\\ \\
\textit{Group3}\\
$\begin{aligned}
    f_1(n) &= n^{\sqrt{n}} \\
    f_2(n) &= 2^n \\
    f_3(n) &= n^{10}.2^{\frac{n}{2}} \\
    f_4(n) &= \sum_{i=1}^{n} (i+1) \\
\end{aligned}$ \\
\text{Ta có:} \\
$\begin{aligned}
    f_1(n) &= n^{\sqrt{n}} = (2^{\log_2{n}})^{\sqrt{n}} = 2^{n^{\frac{1}{2}}.\log_2{n}} = 2^{O(n^{c + \frac{1}{2}})} \quad (\text{c có thể rất nhỏ})\\
    f_2(n) &= 2^n = 2^{O(n)} \\
    f_3(n) &= n^{10}.2^{\frac{n}{2}} = (2^{log_2{n}})^{10}.2^{\frac{n}{2}} = 2^{10.\log_2{n} + \frac{n}{2}} = 2^{O(n)} \\
    f_4(n) &= \sum_{i=1}^{n} (i+1) = \dfrac{n(n+1)}{2} + n = O(n^2) \\
\end{aligned}$ \\ \\
$\Rightarrow f_1(n) < f_2(n) = f_3(n) \quad (\text{vì cùng cơ số 2 và } O(n^{c + \frac{1}{2}}) < O(n))$\\
\text{Mà } $f_4(n) < f_1(n) \quad (\text{vì } n^2 < n^{\sqrt{n}})$\\
$\Rightarrow f_4(n) < f_1(n) < f_2(n) = f_3(n)$\\ \\
{\textit{Group4}} \\
$\begin{aligned}
    f_6(n) &= n^{\sqrt{n}} = (2^{\log_2{n}})^{\sqrt{n}} = 2^{n^{\frac{1}{2}}.\log_2{n}} = 2^{O(n^{c + \frac{1}{2}})} \quad (\text{c có thể rất nhỏ})\\
    f_7(n) &= \pi^n = (2^{\log_2{\pi}})^n = 2^{n.\log_2{\pi}} = 2^{O(n)}\\
    f_8(n) &= 2^{n^4} = 2^{O(n^4)} \\
    f_9(n) &= n^{4logn} = (2^{\log_2{n}})^{4\log_2{n}} = 2^{4\log_2{n}.\log_2{n}} = 2^{O(n^{2c})} \quad (\text{c có thể rất nhỏ})\\
\end{aligned}$ \\ \\
$\Rightarrow f_9(n) < f_6(n) < f_7(n) < f_8(n) \\
(\text{vì cùng cơ số 2 và } O(n^{2c}) < O(n^{c + \frac{1}{2}}) < O(n) < O(n^4))$\\ \\
\textit{Group5}\\
$\begin{aligned}
    f_6(n) &= n^{\sqrt{n}} = (2^{\log_2{n}})^{\sqrt{n}} = 2^{n^{\frac{1}{2}}.\log_2{n}} = 2^{O(n^{c + \frac{1}{2}})} \quad (\text{c có thể rất nhỏ})\\
    f_7(n) &= n^{\log n} = (2^{\log_2{n}})^{\log_2{n}} = 2^{\log_2{n}.\log_2{n}} = 2^{O(n^{2c})}  \quad (\text{c có thể rất nhỏ})\\
    f_8(n) &= 2^{\frac{n}{2}} = 2^{O(n)} \\
    f_9(n) &= 3^{\sqrt{n}} = (2^{\log_2{3}})^{\sqrt{n}} = 2^{n^{\frac{1}{2}}.\log_2{3}} = 2^{O(n^{\frac{1}{2}})}\\
    f_{10}(n) &= 4^{n^{\frac{1}{4}}} = (2^2)^{n^{\frac{1}{4}}} = 2^{2.n^{\frac{1}{4}}} = 2^{O(n^{\frac{1}{4}})}\\
\end{aligned}$ \\
$\Rightarrow f_7(n) < f_6(n) < f_{10}(n) < f_9(n) < f_8(n)$ \\
$(\text{vì cùng cơ số 2 và } O(n^{2c}) < O(n^{c + \frac{1}{2}}) < O(n^{\frac{1}{4}}) < O(n^{\frac{1}{2}}) < O(n))$\\ \\
\textit{Group6}\\
$\begin{aligned}
    f_1(n) &= n^{0.999999}logn = O(n^{0.999999 + c}) \quad (\text{c có thể rất nhỏ}) \\
    f_2(n) &= 10000000n = O(n) \\
    f_3(n) &= 1.000001^n = O(1.000001^n) \\
    f_4(n) &= n^2 = O(n^2) \\
\end{aligned}$ \\
$\Rightarrow f_1(n) < f_2(n) < f_4(n) < f_3(n) \quad (\text{vì } O(n^{0.999999 + c}) < O(n) < O(n^2) < O(a^n) \text{ với a > 1})$\\ \\
\textit{Group7}\\
$\begin{aligned}
    f_1(n) &= n^{\pi} \\
    f_2(n) &= \pi^n \\
    f_3(n) &= \binom{n}{5} \\\
    f_4(n) &= \sqrt{2^{\sqrt{n}}} \\
    f_5(n) &= \binom{n}{n-4} \\
    f_6(n) &= 2^{log^4n} \\
    f_7(n) &= n^{5.(logn)^2} \\
    f_8(n) &= n^4 \binom{n}{4} \\
\end{aligned}$ \\
\text{Ta có:} \\
$\begin{aligned}
    f_1(n) &= n^{\pi} = O(n^{\pi}) \\
    f_3(n) &= \binom{n}{5} = \dfrac{n!}{5!(n-5)!} = \dfrac{n(n-1)(n-2)(n-3)(n-4)}{5!} = O(n^5) \\
    f_5(n) &= \binom{n}{n-4} = \dfrac{n!}{(n-4)!4!} = \dfrac{n(n-1)(n-2)(n-3)}{4!} = O(n^4) \\
    f_8(n) &= n^4 \binom{n}{4} = n^4.\dfrac{n!}{4!(n-4)!} = n^4.\dfrac{n(n-1)(n-2)(n-3)}{4!} = O(n^8) \\
\end{aligned}$ \\
$\Rightarrow f_1(n) < f_5(n) < f_3(n) < f_8(n) \quad (\text{vì } O(n^{\pi}) < O(n^4) < O(n^5) < O(n^8))$\\ \\
$\begin{aligned}
    f_2(n) &= \pi^n = (2^{\log_2{\pi}})^n = 2^{n.\log_2{\pi}} = 2^{O(n)} \\
    f_4(n) &= \sqrt{2^{\sqrt{n}}} = (2^{\sqrt{n}})^{\frac{1}{2}} = 2^{n^{\frac{1}{2}}.\frac{1}{2}} = 2^{O(n^{\frac{1}{2}})} \\
    f_6(n) &= 2^{log^4n} = 2^{O(n^{4c})} \quad (\text{c có thể rất nhỏ}) \\
    f_7(n) &= n^{5.(logn)^2} = (2^{\log_2{n}})^{5.(\log_2{n})^2} = 2^{5.(\log_2{n})^3} = 2^{O(n^{3c})} \quad (\text{c có thể rất nhỏ})\\
\end{aligned}$ \\
$\Rightarrow f_7(n) < f_6(n) < f_4(n) < f_2(n) \\
(\text{vì cùng cơ số 2 và } O(n^{3c}) < O(n^{4c}) < O(n^{\frac{1}{2}}) < O(n))$\\ \\
\text{Mà } $f_8(n) < f_7(n) \quad (\text{vì 8 chỉ là một hằng số còn $logn$ phụ thuộc vào n nên } n^8 < n^{5.(logn)^2})$\\
$\Rightarrow f_1(n) < f_5(n) < f_3(n) < f_8(n) < f_7(n) < f_6(n) < f_4(n) < f_2(n)$\\ \\
\section{Chứng minh, dùng định nghĩa của các ký hiệu tiệm cận (không dùng lim)}
\begin{enumerate}[label=\bfseries\large\alph*.]
    \item $n^4 + n + 1 \notin O(n^2)$ \\ \\
    \text{Ta có:} \\
    $n^4 + n + 1 > n^2 \quad \forall n \geq 1$ \\
    $\Rightarrow$ \text{Không tồn tại 1 hằng số dương c nào thỏa} \\
    $n^4 + n + 1 \leq c.n^2 \quad \forall n \geq n_0 (n_0 \in \mathbb{N})$
    $\text{Vậy } n^4 + n + 1 \notin O(n^2)$

    \item $O(c.f(n)) = O(f(n)) \text{ với c là hằng số}$ \\ \\
    \text{- Xét một hàm bất kỳ g(n) $\in O(c.f(n))$} \\
    $\Rightarrow \exists c_1 \in \mathbb{R^+}, n_0 \in \mathbb{N} \text{ sao cho:}$ \\
    $g(n) \leq c_1.(cf(n)) \quad \forall n \geq n_0$ \\
    $g(n) \leq (c_1.c).f(n)$ \\
    $\Rightarrow \exists d = c_1.c \in \mathbb{R^+},n_1 = n_0 \in \mathbb{N} \text{ sao cho:}$ \\
    $g(n) \leq d.f(n) \quad \forall n \geq n_1$ \\
    $\Rightarrow g(n) \in O(f(n)) \quad (1)$ \\

    \text{- Xét một hàm bất kỳ g(n) $\in O(f(n))$} \\
    $\Rightarrow \exists c_1 \in \mathbb{R^+}, n_0 \in \mathbb{N} \text{ sao cho:}$ \\
    $g(n) \leq c_1.f(n) \quad \forall n \geq n_0$ \\
    $g(n) \leq \dfrac{c_1}{c}.(c.f(n))$ \\
    $\Rightarrow \exists d = \dfrac{c_1}{c} \in \mathbb{R^+},n_1 = n_0 \in \mathbb{N} \text{ sao cho:}$ \\
    $g(n) \leq d.(cf(n)) \quad \forall n \geq n_1$ \\
    $\Rightarrow g(n) \in O(c.f(n)) \quad (2)$ \\

    \text{Từ (1) và (2) ta có:} $O(c.f(n)) = O(f(n))$\\
    \text{Vậy } $O(c.f(n)) = O(f(n))$

    \item $\text{Nếu } f(n) \in O(g(n)) \text{ và } g(n) \in O(h(n)) \text{ thì } f(n) \in O(h(n))$ \\ \\
    \text{- Xét $f(n) \in O(g(n))$} \\
    $\Rightarrow \exists c \in \mathbb{R^+}, n_0 \in \mathbb{N} \text{ sao cho:}$ \\
    $f(n) \leq c.g(n) \quad \forall n \geq n_0 \quad (1)$ \\

    \text{- Xét $g(n) \in O(h(n))$} \\
    $\Rightarrow \exists c_1 \in \mathbb{R^+}, n_1 \in \mathbb{N} \text{ sao cho:}$ \\
    $g(n) \leq c_1.h(n) \quad \forall n \geq n_1 \quad (2)$ \\

    \text{Từ (1) và (2) ta có:} $f(n) \leq (c.c_1)h(n)$\\
    $\Rightarrow \exists d = c.c_1 \in \mathbb{R^+},n_2 = \max\{n_0, n_1\} \in \mathbb{N} \text{ sao cho:}$ \\
    $f(n) \leq d.h(n) \quad \forall n \geq n_2$ \\
    $\Rightarrow f(n) \in O(h(n))$\\
    \text{Vậy giả thiết ở câu c đã được chứng minh} \\
    \item $\max{\{f(n), g(n)\}} = \Theta(f(n)+g(n))$ \\\\
		$\dfrac{f(n) + g(n)}{2} \leq \max{\{f(n), g(n)\}} \leq f(n) + g(n) \quad\forall n \geq n_0$ \\
		Chọn $c_1 = \dfrac{1}{2}, c_2 = 1, n_0 = 1$, theo định nghĩa của $Big-\Theta$, ta có điều cần chứng minh: $\max{\{f(n), g(n)\}} = \Theta(f(n)+g(n))$
		\item $g(n) \in O(h(n)) \Rightarrow O(g(n)) \subseteq O(h(n))$ \\\\
		Xét 1 hàm bất kỳ $f(n) \in O(g(n))$ \\
		$\Leftrightarrow \exists c_1 \in \mathbb{R^+}, n_1 \in \mathbb{N}$ sao cho:
		$f(n) \leq c_1.g(n) \quad\forall n \geq n_1 \quad (1)$ \\
		Ta có: $g(n) \in O(h(n))$ \\
		$\Leftrightarrow \exists c_2 \in \mathbb{R^+}, n_2 \in \mathbb{N}$ sao cho:
		$g(n) \leq c_2.h(n) \quad\forall n \geq n_2 \quad (2)$ \\
		Đặt: $n_0 = \max{\{n_1, n_2\}}$ \\
		Từ (1), (2) $\Rightarrow f(n) \leq c_1c_2.h(n) \quad\forall n \geq n_0
		\quad \Leftrightarrow \quad f(n) \in O(h(n))$ \\
		$\Rightarrow$ Mọi phần tử của O(g(n)) đều thuộc O(h(n)) $\Rightarrow O(g(n)) \subseteq O(h(n))$ \\
		Vậy: $g(n) \in O(h(n)) \Rightarrow O(g(n)) \subseteq O(h(n))$
		\item $\Theta(g(n)) = O(g(n)) \cap \Omega(g(n))$ \\\\
		Xét 1 hàm bất kỳ $f(n) \in \Theta(g(n))$ \\
		$\Leftrightarrow \exists c_1, c_2 \in \mathbb{R^+}, n_0 \in \mathbb{N}$ sao cho: 
		$c_1.g(n) \leq f(n) \leq c_2.g(n) \quad\forall n \geq n_0$
		\begin{gather*}
			\Leftrightarrow\left\{\begin{array}{lll}
				f(n) \geq c_1.g(n) \\
				f(n) \leq c_2.g(n) 
			\end{array}\right. \forall n \geq n_0
			\quad 
			\Leftrightarrow
			\hspace{0.6cm}
			\left\{\begin{array}{lll}
				f(n) = \Omega(g(n)) \\
				f(n) = O(g(n))
			\end{array}\right.
		\end{gather*}
		Vậy ta có điều cần chứng minh: $\Theta(g(n)) = O(g(n)) \cap \Omega(g(n))$.
		\item $n + n^2O(lnn) = O(n^2lnn)$ \\\\
		Xét 1 hàm bất kỳ $f(n) \in O(lnn)$ \\
		$\Rightarrow \exists c_1 \in \mathbb{R^+}, n_0 \in \mathbb{N}$ sao cho:
		$f(n) \leq c_1lnn \quad\forall n\geq n_0$ \\
		$\Leftrightarrow n^2.f(n) \leq c_1n^2lnn \quad\forall n\geq n_0$ \\
		$\Leftrightarrow n + n^2.f(n) \leq n + c_1n^2lnn \quad\forall n\geq n_0$ \\
		Xét: $n + c_1n^2lnn \leq c_2n^2lnn \quad\forall n\geq n_0$ (với $c_2 \in \mathbb{R^+}$) \\
		$\Leftrightarrow 1 + c_1nlnn \leq c_2nlnn \quad\forall n\geq n_0$ \\
		$\Leftrightarrow 1 \leq (c_2 - c_1)nlnn \quad\forall n\geq n_0$ \\
		Chọn $c_1 = 1, c_2 = 2, n_0 =2$, theo định nghĩa của Big-O ta có điều cần chứng minh.
\end{enumerate}
\section{Các khẳng định bên dưới là đúng hay sai? Vì sao?}
	\begin{enumerate}
		\item Nếu f(n) = $\Theta$(g(n)) và g(n) = $\Theta$(h(n)), thì h(n) = $\Theta$(f(n)) \\ \\
		Ta có: f(n) = $\Theta$(g(n)) \\
		$\Rightarrow \exists a_1, a_2 \in \mathbb{R^+}, n_a \in \mathbb{N} \text{ sao cho:}$ 
		$a_1.g(n) \leq f(n) \leq a_2.g(n) \quad \forall n \geq n_a \quad (1)$ \\ \\
		Ta có: g(n) = $\Theta$(h(n)) \\
		$\Rightarrow \exists b_1, b_2 \in \mathbb{R^+}, n_b \in \mathbb{N} \text{ sao cho:}$ 
		$b_1.h(n) \leq g(n) \leq b_2.h(n) \quad \forall n \geq n_b$ \\
		$\Rightarrow \dfrac{1}{b_2}.g(n) \leq h(n) \leq \dfrac{1}{b_1}.g(n) \quad\forall n \geq n_b \qquad\qquad\qquad\qquad\qquad\qquad\qquad (2)$ \\ \\
		$\text{Đặt: } n_0 = max(n_a, n_b)$ \\
		$\text{Từ (1) } \Rightarrow \dfrac{1}{a_2}.f(n) \leq g(n) \leq \dfrac{1}{a_1}.f(n) \quad\forall n \geq n_a$ \\
		$\Rightarrow \dfrac{1}{b_1}.g(n) \leq \dfrac{1}{a_1b_1}.f(n) \text{ và } \dfrac{1}{a_2b_2}.f(n) \leq \dfrac{1}{b_2}.g(n) \quad\forall n \geq n_0 \qquad\quad\quad\quad\quad (3)$ \\
		$\text{Từ (2) và (3) } \Rightarrow \dfrac{1}{a_2b_2}.f(n) \leq h(n) \leq \dfrac{1}{a_1b_1}.f(n) \quad\forall n \geq n_0$ \\ \\
		$\text{Đặt: } c_1 = \dfrac{1}{a_1b_1} \text{ và } c_2 = \dfrac{1}{a_2b_2}$ \\
		$\text{Suy ra: } \exists c_1, c_2 \in \mathbb{R^+}, n_0 \in \mathbb{N} \text{ sao cho:}$
		$c_1.f(n) \leq h(n) \leq c_2.f(n) \quad \forall n \geq n_0$ \\ 
		$\Leftrightarrow h(n) = \Theta(f(n))$ \\
		Vậy: khẳng định này là đúng.
		\item Nếu f(n) = O(g(n)) và g(n) = O(f(n)), thì f(n) = g(n) \\ \\
		(Các tính chất trong câu này có trong file C1T6 trang 22) \\
		Ta có tính chất:  f(n) = O(g(n)) và g(n) = O(f(n)) $\Leftrightarrow$ O(f(n)) = O(g(n)) \\
		Xét ví dụ: f(n) = n, g(n) = 2n \qquad\qquad\qquad\qquad\qquad(1) \\
		Ta có tính chất: O(Cf(n)) = O(f(n)) với C là hằng số (2) \\
		Từ (1), (2) suy ra: O(f(n)) = O(g(n)) nhưng f(n) $\ne$ g(n) \\
		Tức là: f(n) = O(g(n)) và g(n) = O(f(n)), thì có thể f(n) $\ne$ g(n) \\
		Vậy: khẳng định này là sai.
		\item f(n) + O(f(n)) = $\Theta$(f(n)) \\\\
		$\text{Xét 1 hàm bất kỳ } g(n) \in O(f(n))$ \\
		$\Rightarrow \exists a \in \mathbb{R^+}, n_0 \in \mathbb{N} \text{ sao cho:} $
		$g(n) \leq a.f(n) \quad\forall n \geq n_0$ \\
		$\Rightarrow 1.f(n) \leq f(n) + g(n) \leq (a+1)f(n) \quad (\text{Vì g(n) là hàm chi phí nên không thể âm})$ \\
		Đặt c$_1$ = 1, c$_2$ = a+1 \\
		Suy ra: $c_1f(n) \leq f(n) + g(n) \leq c_2f(n) \quad\forall n \geq n_0$ \\
		$\Rightarrow f(n) + O(f(n)) = \Theta(f(n))$ \\
		Vậy: khẳng định này là đúng.
		\item 2$^{10n}$ = O(2$^n$) \\\\
		Giả sử: $2^{10n} = O(2^n)$ \\
		$\Rightarrow \exists c \in \mathbb{R^+}, n_0 \in \mathbb{N} \text{ sao cho:}$
		$2^{10n} \leq c.2^n \quad\forall n \geq n_0$ \\
		$\Leftrightarrow 2^{9n} \leq c \Leftrightarrow n \leq \dfrac{1}{9}.\log_2c \quad\forall n \geq n_0$ \\
		Đặt: $\alpha = \dfrac{1}{9}.\log_2c \Rightarrow n \leq \alpha \quad\forall n \geq n_0$ 
		\begin{itemize}
		\item Xét trường hợp 1: $\alpha < n_0 \quad$ \\\\
		\begin{tikzpicture}[scale=1] 
			\draw[->,thin] (-5,0)--(5,0); 
			\foreach \x in {-3}\fill[black] (\x,0) circle (1pt)node[below]{$\alpha$};
			\foreach \x in {-1}\fill[black] (\x,0) circle (1pt)node[below]{$n_0$};
			\foreach \x in {2}\fill[black] (\x,0) circle (1pt)node[below]{$n$};
		\end{tikzpicture} \\
		Với $n \geq n_0$ thì $n > \alpha \Rightarrow$ Mâu thuẫn. \\
		\item Xét trường hợp 2: $\alpha \geq n_0$ \\\\
		\begin{tikzpicture}[scale=1] 
			\draw[->,thin] (-5,0)--(5,0); 
			\foreach \x in {-3}\fill[black] (\x,0) circle (1pt)node[below]{$n_0$};
			\foreach \x in {-1}\fill[black] (\x,0) circle (1pt)node[below]{$\alpha$};
			\foreach \x in {2}\fill[black] (\x,0) circle (1pt)node[below]{$n$};
		\end{tikzpicture} \\
		Sẽ có những trường hợp $n > \alpha$ như trên trục số $\Rightarrow$ Mâu thuẫn. 
		\end{itemize}
		Vậy: Giả sử sai $\Rightarrow$ Khẳng định này là sai.
		\item 2$^{n+10}$ = O(2$^n$) \\\\
		Ta có: $2^{n+10} = 2^{10}.2^n \leq 2^{10}.2^n \quad\forall n \geq 1$ \\
		Chọn $c = 2^{10}$, $n_0 = 1$, theo định nghĩa của Big-O, ta có: 2$^{n+10}$ = O(2$^n$)\\
		Vậy: khẳng định này là đúng.
		\item $\log_an$ = $\Theta$($\log_bn$) \\\\
		Điều kiện xác định: a>1, b>1 (bởi vì $n \in \mathbb{N}$ nên a, b không thể nhỏ hơn 1) \\
		Giả sử: $\log_an$ = $\Theta$($\log_bn$) \\
		$\Rightarrow \exists c_1, c_2 \in \mathbb{R^+}, n_0 \in \mathbb{N}$ sao cho: 
		\begin{center}
		$c_1.\log_bn \leq \log_an \leq c_2.\log_bn \quad\forall n \geq n_0$ \\
		$\Leftrightarrow c_1.\log_bn \leq \log_ab.\log_bn \leq c_2.\log_bn \quad\forall n \geq n_0$ \\
		$\Leftrightarrow c_1 \leq \log_ab \leq c_2 \quad\forall n \geq n_0$
		\end{center}
		Luôn tồn tại $c_1, c_2$ và $n_0 > 1$ thoải mãn điều kiện trên $\Rightarrow$ Giả sử đúng. \\
		Vậy: khẳng định này là đúng.
	\end{enumerate}
% 5
\section{Ước lượng nhanh độ phức tạp của giải thuật đệ quy dùng Định lý Master}
\begin{enumerate}[label=\bfseries\large\theenumi.]
    \item $ T(n) = 3T\left( \dfrac{n}{2} \right) + n^2 $ (Dạng đơn giản 1) \\ \\
        $ a = 3, b = 2, d = 2 $ \\
        $ \rightarrow \text{Vì } 3 < 2^2 \text{ nên trường hợp 1 được áp dụng} $ \\
        $ T(n) \in \Theta(n^d) = \Theta(n^2) $

    \item $ T(n) = 7T\left( \dfrac{n}{3} \right) + n^2 $ (Dạng đơn giản 1) \\ \\
        $ a = 7, b = 3, d = 2 $ \\
        $ \rightarrow \text{Vì } 7 < 3^2 \text{ nên trường hợp 1 được áp dụng} $ \\
        $ T(n) \in \Theta(n^d) = \Theta(n^2) $

    \item $ T(n) = 3T\left( \dfrac{n}{3} \right) + \dfrac{n}{2} $ (Dạng đơn giản 1) \\ \\
        $ a = 3, b = 3, d = 1 $ \\
        $ \rightarrow \text{Vì } 3 = 3^1 \text{ nên trường hợp 2 được áp dụng} $ \\
        $ T(n) \in \Theta(n^d\log n) = \Theta(n\log n) $

    \item $ T(n) = 16T\left( \dfrac{n}{4} \right) + n $ (Dạng đơn giản 1) \\ \\
        $ a = 16, b = 4, d = 1 $ \\
        $ \rightarrow \text{Vì } 16 > 4^1 \text{ nên trường hợp 3 được áp dụng} $ \\
        $ T(n) \in \Theta(n^{\log_b a}) = \Theta(n^2) $

    \item $ T(n) = 2T\left( \dfrac{n}{4} \right) + n^{0.51} $ (Dạng đơn giản 1) \\ \\
        $ a = 2, b = 4, d = 0.51 $ \\
        $ \rightarrow \text{Vì } 2 < 4^{0.51} \text{ nên trường hợp 1 được áp dụng} $ \\
        $ T(n) \in \Theta(n^d) = \Theta(n^{0.51}) $

    \item $ T(n) = 3T\left( \dfrac{n}{2} \right) + n $ (Dạng đơn giản 1) \\ \\
        $ a = 3, b = 2, d = 1 $ \\
        $ \rightarrow \text{Vì } 3 > 2^1 \text{ nên trường hợp 3 được áp dụng} $ \\
        $ T(n) \in \Theta(n^{\log_b a}) = \Theta(n^{\log_2 3}) $

    \item $ T(n) = 3T\left( \dfrac{n}{3} \right) + \sqrt{(n)} $ (Dạng đơn giản 1) \\ \\
        $ a = 3, b = 3, d = \frac{1}{2} $ \\
        $ \rightarrow \text{Vì } 3 > 3^{\frac{1}{2}} \text{ nên trường hợp 3 được áp dụng} $ \\
        $ T(n) \in \Theta(n^{\log_b a}) = \Theta(n) $

    \item $ T(n) = 4T\left( \dfrac{n}{2} \right) + cn $ (Dạng đơn giản 1) \\ \\
        $ a = 4, b = 2, d = 1 $ \\
        $ \rightarrow \text{Vì } 4 > 2^1 \text{ nên trường hợp 3 được áp dụng} $ \\
        $ T(n) \in \Theta(n^{\log_2 4}) = \Theta(n^2) $

    \item $ T(n) = 4T\left( \dfrac{n}{4} \right) + 5n $ (Dạng đơn giản 1) \\ \\
        $ a = 4, b = 4, d = 1 $ \\
        $ \rightarrow \text{Vì } 4 = 4^1 \text{ nên trường hợp 2 được áp dụng} $ \\
        $ T(n) \in \Theta(n^d \log n) = \Theta(n \log n) $

    \item $ T(n) = 5T\left( \dfrac{n}{4} \right) + 4n $ (Dạng đơn giản 1) \\ \\
        $ a = 5, b = 4, d = 1 $ \\
        $ \rightarrow \text{Vì } 5 > 4^1 \text{ nên trường hợp 3 được áp dụng} $ \\
        $ T(n) \in \Theta(n^{\log_b a}) = \Theta(n^{\log_4 5}) $

    \item $ T(n) = 4T\left( \dfrac{n}{5} \right) + 5n $ (Dạng đơn giản 1) \\ \\
        $ a = 4, b = 5, d = 1 $ \\
        $ \rightarrow \text{Vì } 4 < 5^1 \text{ nên trường hợp 1 được áp dụng} $ \\
        $ T(n) \in \Theta(n^d) = \Theta(n) $

    \item $ T(n) = 25T\left( \dfrac{n}{5} \right) + n^2 $ (Dạng đơn giản 1) \\ \\
        $ a = 25, b = 5, d = 2 $ \\
        $ \rightarrow \text{Vì } 25 = 5^2 \text{ nên trường hợp 2 được áp dụng} $ \\
        $ T(n) \in \Theta(n^d \log n) = \Theta(n^2 \log n) $

    \item $ T(n) = 10T\left( \dfrac{n}{3} \right) + 17n^{1.2} $ (Dạng đơn giản 1) \\ \\
        $ a = 10, b = 3, d = 1.2 $ \\
        $ \rightarrow \text{Vì } 10 > 3^{1.2} \text{ nên trường hợp 3 được áp dụng} $ \\
        $ T(n) \in \Theta(n^{\log_b a}) = \Theta(n^{\log_3 10}) $

    \item $ T(n) = 7T\left( \dfrac{n}{2} \right) + n^3 $ (Dạng đơn giản 1) \\ \\
        $ a = 7, b = 2, d = 3 $ \\
        $ \rightarrow \text{Vì } 7 < 2^3 \text{ nên trường hợp 1 được áp dụng} $ \\
        $ T(n) \in \Theta(n^d) = \Theta(n^3) $

    \item $ T(n) = 4T\left( \dfrac{n}{2} \right) + \log n $ (Dạng tổng quát 2) \\ \\
        $ a = 4, b = 2 \text{ và } n^{\log_b a} = n^{\log_2 4} = n^2 $ \\
        $ \text{Ta thấy: } f(n) = \log n = O(n^c) = O(n^{2 - \epsilon}) (\epsilon > 0) $ \\
        $ \rightarrow \text{Áp dụng trường hợp 1 ta được: } T(n) = \Theta(n^2) $

    \item $ T(n) = 4T\left( \dfrac{n}{5} \right) + \log n $ (Dạng tổng quát 2) \\ \\
        $ a = 4, b = 5 \text{ và } n^{\log_b a} = n^{\log_5 4} $ \\
        $ \text{Ta thấy: } f(n) = \log n = O(n^c) = O(n^{\log_5 4 - \epsilon}) (\epsilon > 0) $ \\
        $ \rightarrow \text{Áp dụng trường hợp 1 ta được: } T(n) = \Theta(n^{\log_5 4}) $

    \item $ T(n) = \sqrt{2} T\left( \dfrac{n}{2} \right) + \log n $ (Dạng tổng quát 2) \\ \\
        $ a = \sqrt{2}, b = 2 \text{ và } n^{\log_b a} = n^{\log_2 \sqrt{2}} = n^{0.5} $ \\
        $ \text{Ta thấy: } f(n) = \log n = O(n^c) = O(n^{0.5 - \epsilon}) (\epsilon > 0) $ \\
        $ \rightarrow \text{Áp dụng trường hợp 1 ta được: } T(n) = \Theta(n^{0.5}) $

    \item $ T(n) = 2T\left( \dfrac{n}{3} \right) + n \log n $ (Dạng tổng quát 2) \\ \\
        $ a = 2, b = 3 \text{ và } n^{\log_b a} = n^{\log_3 2} $ \\
        $ \text{Ta thấy: } f(n) = n \log n = \Omega(n^c) = \Omega(n^{\log_3 2 + \epsilon}) (\epsilon > 0) $ \\
        $ \text{và } af\left(\dfrac{n}{b}\right) \leq cf(n) $ \\
        $ \text{vì } 2\dfrac{n}{3} \log \dfrac{n}{3} = \dfrac{2n}{3}(\log n - \log 3) \leq c n \log n \text{ với } c = \dfrac{2}{3} $ \\
        $ \rightarrow \text{Áp dụng trường hợp 3 ta được: } T(n) = \Theta(f(n)) = \Theta(n \log n) $

    \item $ T(n) = 3T\left( \dfrac{n}{4} \right) + n \log n $ (Dạng tổng quát 2) \\ \\
        $ a = 3, b = 4 \text{ và } n^{\log_b a} = n^{\log_4 3} $ \\
        $ \text{Ta thấy: } f(n) = n \log n = \Omega(n^c) = \Omega(n^{\log_4 3 + \epsilon}) (\epsilon > 0) $ \\
        $ \text{và } af\left(\dfrac{n}{b}\right) \leq cf(n) $ \\
        $ \text{vì } 3\dfrac{n}{4} \log \dfrac{n}{4} = \dfrac{3n}{4}(\log n - \log 4) \leq c n \log n \text{ với } c = \dfrac{3}{4} $ \\
        $ \rightarrow \text{Áp dụng trường hợp 3 ta được: } T(n) = \Theta(f(n)) = \Theta(n \log n) $

    \item $ T(n) = 6T\left( \dfrac{n}{3} \right) + n^2 \log n $ (Dạng tổng quát 2) \\ \\
        $ a = 6, b = 3 \text{ và } n^{\log_b a} = n^{\log_3 6} $ \\
        $ \text{Ta thấy: } f(n) = n^2 \log n = \Omega(n^c) = \Omega(n^{\log_3 6 + \epsilon}) (\epsilon > 0) $ \\
        $ \text{và } af\left(\dfrac{n}{b}\right) \leq cf(n) $ \\
        $ \text{vì } 6{(\dfrac{n}{3})}^2 \log \dfrac{n}{3} = \dfrac{2n^2}{3}(\log n - \log 3) \leq c n^2 \log n \text{ với } c = \dfrac{2}{3} $ \\
        $ \rightarrow \text{Áp dụng trường hợp 3 ta được: } T(n) = \Theta(f(n)) = \Theta(n^2 \log n) $

    \item $ T(n) = 3T\left( \dfrac{n}{5} \right) + \log^2 n $ (Dạng tổng quát 2) \\ \\
        $ a = 3, b = 5 \text{ và } n^{\log_b a} = n^{\log_5 3} $ \\
        $ \text{Ta thấy: } f(n) = \log^2 n = O(n^c) = O(n^{\log_5 3 - \epsilon}) (\epsilon > 0) $ \\
        $ \rightarrow \text{Áp dụng trường hợp 1 ta được: } T(n) = \Theta(n^{\log_5 3}) $

    \item $ T(n) = 2T\left( \dfrac{n}{2} \right) + \dfrac{n}{\log n} $ (Dạng tổng quát 2) \\ \\
        $ a = 2, b = 2 \text{ và } n^{\log_b a} = n^{\log_2 2} = n $ \\
        $ \text{Ta thấy: } f(n) = \dfrac{n}{\log n} = \Theta(n \log^{-1} n) $ \\
        Ta thấy có dạng giống ở trường hợp 2 nhưng vì ở đây $ k = -1 < 0 $ nên cần tham khảo thêm
        một phần mở rộng hữu dụng của định luật Master có thể tìm thấy ở đường dẫn
        \url{https://en.wikipedia.org/wiki/Master_theorem_(analysis_of_algorithms)#Generic_form}
        ở trong bảng thứ 2. \\
        $ \rightarrow \text{Áp dụng trường hợp 2b ta được: } T(n) = \Theta(n^{\log_2 2} \log\log n) = \Theta(n\log\log n) $

    \item $ T(n) = 2^n T\left( \dfrac{n}{2} \right) + n^n $ \\ \\
        Không áp dụng được định lý Master. \\
        Vì $ a = 2^n $ không thể được biểu diễn như một hằng số \\

    \item $ T(n) = 0.5 T\left( \dfrac{n}{2} \right) + n $ \\ \\
        Không áp dụng được định lý Master. Vì $ a = 0.5 < 1 $ \\

    \item $ T(n) = T\left( \dfrac{n}{2} \right) + n(2 - \cos n) $ \\ \\
        $ a = 1, b = 2 \text{ và } n^{\log_b a} = n^{\log_2 1} = n^0 $ \\
        $ \text{Ta thấy: } f(n) = n (2 - \cos n) = 2n - 2 \cos n = \Omega(n^\epsilon) (\epsilon > 0) $ \\
        $ \text{Nhưng vì } \nexists c < 1 \text{ sao cho } af\left(\dfrac{n}{b}\right) \leq cf(n) $
        nên không thể áp dụng được định lý Master.

    \item $ T(n) = 64T\left( \dfrac{n}{8} \right) - n^2 \log n $ \\ \\
        Không áp dụng được định lý Master. \\
        Vì $ f(n) = -n^2 \log n $ không tiến đến dương khi tiệm cận \\

    \item $ T(n) = T\left( \dfrac{n}{2} \right) + 2^n $ (Dạng tổng quát 2) \\ \\
        $ a = 1, b = 2 \text{ và } n^{\log_b a} = n^{\log_2 1} = n^0 $ \\
        $ \text{Ta thấy: } f(n) = 2^n = \Omega(n^\epsilon) (\epsilon > 0) $ \\
        $ \text{và } af\left(\dfrac{n}{b}\right) \leq cf(n) $ \\
        $ \text{vì } 2^{\frac{n}{2}} \leq c 2^n, c = 0.9 $ \\
        $ \rightarrow \text{Áp dụng trường hợp 3 ta được: } T(n) = \Theta(f(n)) = \Theta(2^n) $

    \item $ T(n) = 16T\left( \dfrac{n}{4} \right) + n! $ (Dạng tổng quát 2)\\ \\
        $ a = 16, b = 4 \text{ và } n^{\log_b a} = n^{\log_4 16} = n^2 $ \\
        $ \text{Ta thấy: } f(n) = n! = \Omega(n^{2 + \epsilon}) (\epsilon > 0) $ \\
        $ \text{nhưng vì } \nexists c < 1 \text{ sao cho } 16 \dfrac{n}{4}! \leq c n! $
        nên không thể áp dụng được định lý Master. \\
\end{enumerate}
\end{document}
