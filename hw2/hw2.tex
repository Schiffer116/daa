\documentclass[12pt, letterpaper]{article}
\usepackage[top = 1cm, bottom = 0.75cm, left = 1in, right = 1in]{geometry}
\usepackage[utf8]{inputenc}
\usepackage[vietnamese]{babel}
\usepackage{graphicx}
\usepackage{amsmath}
\usepackage{titlesec}
\usepackage{color}
\usepackage{xcolor}
\usepackage{listings}
\usepackage{array}
\usepackage{adjustbox}
\usepackage{booktabs}
\geometry{margin=3cm}

\titleformat{\section}
{\Large\bfseries}
{Bài tập \thesection: }
{0em}
{}

\titleformat{\subsection}
{\large\bfseries}
{{\thesubsection} }
{0em}
{}

\renewcommand{\theenumi}{\bfseries\large\alph{enumi}}

\lstset{
  basicstyle=\ttfamily,
  mathescape
}

\title{
  \large\textbf{TRƯỜNG ĐẠI HỌC CÔNG NGHỆ THÔNG TIN} \\
  \large\textbf{KHOA KHOA HỌC MÁY TÍNH} \\
  \vfill
  \begin{figure}[h]
    \centering
    \includegraphics[width=0.5\linewidth]{uit}
  \end{figure}
  \vfill
  \textbf{BÀI TẬP MÔN PHÂN TÍCH VÀ THIẾT KẾ THUẬT TOÁN} \\
  \vspace{1cm}
  \Large \textbf{HOMEWORK \#02:} \\
  \large \text{PHÂN TÍCH THUẬT TOÁN ĐỆ QUY}
  \vfill
}

\author{
  \begin{tabular}{rl}
    \textbf{GV hướng dẫn:} &Huỳnh Thị Thanh Thương \\
    \textbf{Nhóm thực hiện:}
    &1. Nguyễn Gia Bảo 22520108 \\
    &2. Võ Đình Khánh 22520659 \\
    &3. Nguyễn Trần Phúc 22521135 \\
    &4. Hồ Trọng Duy Quang 22521200 \\
  \end{tabular}
}

\date{{\vfill}Thành phố Hồ Chí Minh, \MakeLowercase{\today}}

\begin{document}
\maketitle
\pagebreak
\newgeometry{margin=3cm}

\section{Thành lập phương trình đệ quy, kèm giải thích cách thành lập. Không giải phương trình}
\begin{enumerate}
	\item \text{Gửi ngân hàng 1000 USD, lãi suất 12\%/năm. Số tiền có được sau 30 năm là bao nhiêu?}
	
    \textbf{Lời giải:}
\begin{lstlisting}
double interest_rate(int n) {
    if (n == 0) return 1000;
    return interest_rate(n - 1)*1.12;
}
\end{lstlisting}
    $T(n) =
    \begin{cases}
    1000 &\text{khi n = 0} \\
    T(n-1) &\text{khi n > 0}
    \end{cases}$ \\

    \textit{Giải thích:} \\
    \text{-Khi n = 0, số năm gửi tiền là 0 nên sẽ trả về số tiền ban đầu là 1000 USD.} \\
    $\Rightarrow T(0) = 1000$ \\
    \text{-Khi n > 0, số tiền sau năm thứ n sẽ bằng số tiền sau năm trước đó nhân với lãi suất 1.12.} \\
    $\Rightarrow T(n) = T(n-1)$ \\

    \item \begin{lstlisting}
long Fibo(int n) {
    if(n == 0 || n == 1) return 1;
    return Fibo(n - 1) + Fibo(n - 2);
}
    \end{lstlisting}
    \textbf{Lời giải:} \\ \\
    $T(n) =
    \begin{cases}
    1 &\text{khi n = 0 hoặc n = 1} \\
    T(n-1) + T(n-2) &\text{khi n > 1}
    \end{cases}$ \\ 

    \textit{Giải thích:} \\
    \text{-Khi n = 0 hoặc n = 1, số Fibonacci sẽ trả về 1.} \\
    $\Rightarrow T(0) = T(1) = 1$ \\
    \text{-Khi n > 1, số Fibonacci sẽ bằng tổng của 2 số Fibonacci trước đó.} \\
    $\Rightarrow T(n) = T(n-1) + T(n-2)$ \\

	\item \begin{lstlisting}
public int g(int n) {
    if (n == 1)
        return 2;
    else
        return 3*g(n/2) + g(n/2) + 5;
}
    \end{lstlisting}
    \textbf{Lời giải:} \\ \\
    $T(n) =
    \begin{cases}
    2 &\text{khi n = 1} \\
    2T(n/2) + 5 &\text{khi n > 1}
    \end{cases}$ \\

    \textit{Giải thích:} \\
    \text{-Khi n = 1, hàm sẽ trả về 2.} \\
    $\Rightarrow T(1) = 2$ \\
    \text{-Khi n > 1, hàm sẽ trả về với 2 lần gọi hàm g(n/2) cộng với 5.} \\
    $\Rightarrow T(n) = 2T(n/2) + 5$ \\

	\item \begin{lstlisting}
long xn(int n) {
  if(n == 0) return 1;
  long s = 0;
  for(int i=1; i<=n; i++) {
    s = s + i*i*xn(n-i);
  }
  return s;
}
	\end{lstlisting}
  \textbf{Lời giải:} \\ \\
  $T(n) =
  \begin{cases}
  1 &\text{khi n = 0} \\
  \sum_{i=1}^{n} T(n-i) + n &\text{khi n > 0}
  \end{cases}$ \\

  \textit{Giải thích:} \\
  \text{-Khi n = 0, hàm sẽ trả về 1.} \\
  $\Rightarrow T(0) = 1$ \\
  \text{-Khi n > 0, hàm xn(n-i) sẽ được gọi n lần trong vòng lặp và biến s được gán n lần} \\
  $\Rightarrow T(n) = \sum_{i=1}^{n} T(n-i) + n$ \\


	\item \begin{lstlisting}
Draw(n) {
  if($n < 1$) return 0;
  for(i = 1; i <= n; i++) {
    for(j = 1; j <= n; j++) {
      print("*");
    }
  }
  Draw(n-3);
}
	\end{lstlisting}
  \textbf{Lời giải:} \\ \\
  $T(n) =
  \begin{cases}
  0 &\text{khi n < 1} \\
  T(n-3) + n^2 &\text{khi n $\geq$ 1}
  \end{cases}$ \\

  \textit{Giải thích:} \\
  \text{-Khi n < 1, hàm sẽ trả về 0.} \\
  $\Rightarrow T(n) = 0$ \\
  \text{-Khi n $\geq$ 1, hàm sẽ in ra $n^2$ dấu * và gọi hàm Draw(n-3).} \\
  $\Rightarrow T(n) = T(n-3) + n^2$ \\

	\item \begin{lstlisting}
hanoi(n, A, B, C) {
  if(n == 1) transfer(A, C);
  else {
    hanoi(n-1, A, C, B);
    transfer(A, C);
    hanoi(n-1, B, A, C);
  }
}
  \end{lstlisting}

  \textbf{Lời giải:} \\ \\
  $T(n) =
  \begin{cases}
  1 &\text{khi n = 1} \\
  2T(n-1) + 1 &\text{khi n > 1}
  \end{cases}$ \\

  \textit{Giải thích:} \\
  \text{-Khi n = 1, hàm sẽ thực hiện chuyển đĩa từ cột A sang cột C, coi chi phí của hàm Transfer(A, C) là 1} \\
  $\Rightarrow T(1) = 1$ \\
  \text{-Khi n > 1, hàm sẽ gọi đệ quy hanoi(n-1, A, C, B), hanoi(n-1, B, A, C) và chi phí của hàm Transfer(A, C) là 1} \\
  $\Rightarrow T(n) = 2T(n-1) + 1$ \\

\end{enumerate}
\section{Giải các phương trình đệ quy sau dùng phương pháp truy hồi}
\subsection{$T(n) = 2T(\dfrac{n}{2}) + n^2 \quad \quad T(1) = 1$}
\textbf{Lời giải:} \\
$ \begin{aligned}
    T(n) &= 2T(\dfrac{n}{2}) + n^2 \\
        &= 2[2T(\dfrac{n}{4}) + \dfrac{n^2}{4}] + n^2 \\
        &= 4T(\dfrac{n}{4}) + \dfrac{n^2}{2} + n^2 \\
        &= 4[2T(\dfrac{n}{8}) + \dfrac{n^2}{16}] + \dfrac{n^2}{2} + n^2 \\
        &= 8T(\dfrac{n}{8}) + \dfrac{n^2}{4} + \dfrac{n^2}{2} + n^2 \\
        &\cdots \cdots \\
        &= 2^iT(\dfrac{n}{2^i}) + \dfrac{n^2}{2^{i-1}} + \dfrac{n^2}{2^{i-2}} + \cdots + n^2 \\
        &= 2^iT(\dfrac{n}{2^i}) + n^2(1 + \dfrac{1}{2} + \dfrac{1}{4} + \cdots + \dfrac{1}{2^{i-1}}) \\
        &= 2^iT(\dfrac{n}{2^i}) + n^2 \sum_{j=0}^{i-1} (\dfrac{1}{2})^j \\
        &= 2^iT(\dfrac{n}{2^i}) + n^2 . \dfrac{(\dfrac{1}{2})^i - 1}{\dfrac{1}{2} - 1} \\
        &= 2^iT(\dfrac{n}{2^i}) + 2n^2 [1 - (\dfrac{1}{2})^i] \\
\end{aligned} $ \\ \\ \\
\text{-Quá trình dừng lại khi đạt tới T(1)} \\
$\Rightarrow \dfrac{n}{2^i} = 1$ \\ \\
$\Leftrightarrow i = \log_2{n}$ \\ \\
\text{-Vậy:} \\
$ \begin{aligned}
    T(n) &= 2^{\log_2{n}}T(\dfrac{n}{2^{\log_2{n}}}) + 2n^2[1 - (\dfrac{1}{2})^{\log_2{n}}] \\
        &= nT(1) + 2n^2(1 - \dfrac{1}{n}) \\
        &= n + 2n^2 - 2n \\
        &= 2n^2 - n
\end{aligned} $\\ 

\subsection{$T(n) = 2T(\dfrac{n}{2}) + \log n \quad \quad T(1) = 1$}
\textbf{Lời giải:} \\
$ \begin{aligned}
    T(n) &= 2T(\dfrac{n}{2}) + \log n \\
        &= 2[2T(\dfrac{n}{4}) + \log \dfrac{n}{2}] + \log n \\
        &= 4T(\dfrac{n}{4}) + 2\log \dfrac{n}{2} + \log n \\
        &= 4[2T(\dfrac{n}{8}) + \log \dfrac{n}{4}] + 2\log \dfrac{n}{2} + \log n \\
        &= 8T(\dfrac{n}{8}) + 4\log \dfrac{n}{4} + 2\log \dfrac{n}{2} + \log n \\
        &\cdots \cdots \\
        &= 2^iT(\dfrac{n}{2^i}) + 2^{i-1}\log \dfrac{n}{2^{i-1}} + 2^{i-2}\log \dfrac{n}{2^{i-2}} + \cdots + 2\log \dfrac{n}{2} + \log n \\
        &= 2^iT(\dfrac{n}{2^i}) + 2^{i-1}[\log n - \log 2^{i-1}] + 2^{i-2}[\log n - \log 2^{i-2}] + \cdots + 2(\log n - \log 2) + \log n \\
        &= 2^iT(\dfrac{n}{2^i}) + 2^{i-1}[\log n - (i-1)] + 2^{i-2}[\log n - (i-2)] + \cdots + 2(\log n - 1) + \log n \\
        &= 2^iT(\dfrac{n}{2^i}) + \log n[2^{i-1} + 2^{i-2} + \cdots + 2 + 1] - [2^{i-1}(i-1) + 2^{i-2}(i-2) + \cdots + 2] \\
        &= 2^iT(\dfrac{n}{2^i}) + \log n \sum_{j=0}^{i-1} 2^j - \sum_{j=1}^{i-1} j.2^j \\
        &= 2^iT(\dfrac{n}{2^i}) + \log n. \dfrac{2^i - 1}{2 - 1} - [(i - 1 - 1)2^{i - 1 + 1} + 2] \\
        &= 2^iT(\dfrac{n}{2^i}) + \log n(2^i - 1) - (i-2)2^i - 2 \\
\end{aligned} $ \\ \\ \\
\text{-Quá trình dừng lại khi đạt tới T(1)} \\
$\Rightarrow \dfrac{n}{2^i} = 1$ \\ \\
$\Leftrightarrow i = \log_2{n}$ \\ \\
\text{-Vậy:} \\
$ \begin{aligned}
    T(n) &= 2^{\log_2{n}}T(\dfrac{n}{2^{\log_2{n}}}) + \log n(2^{\log_2{n}} - 1) - (\log_2{n} - 2)2^{\log_2{n}} - 2 \\
        &= nT(1) + \log n(n - 1) - (\log_2{n} - 2)n - 2 \\
        &= n + n\log n - \log n - n\log n + 2n - 2 \\
        &= n + 2n - 2 - \log n\\
        &= 3n - 2 - \log n
\end{aligned} $\\

\subsection{$T(n) = 8T(\dfrac{n}{2}) + n^3 \quad \quad T(1) = 1$}
\textbf{Lời giải:} \\
$ \begin{aligned}
    T(n) &= 8T(\dfrac{n}{2}) + n^3 \\
        &= 8[8T(\dfrac{n}{4}) + \dfrac{n^3}{8}] + n^3 \\
        &= 64T(\dfrac{n}{4}) + n^3 + n^3 \\
        &= 64[8T(\dfrac{n}{8}) + \dfrac{n^3}{64}] + n^3 + n^3 \\
        &= 512T(\dfrac{n}{8}) + n^3 + n^3 + n^3 \\
        &\cdots \cdots \\
        &= 8^iT(\dfrac{n}{2^i}) + in^3 \\
\end{aligned} $ \\ \\ \\
\text{-Quá trình dừng lại khi đạt tới T(1)} \\
$\Rightarrow \dfrac{n}{2^i} = 1$ \\ \\
$\Leftrightarrow i = \log_2{n}$ \\ \\
\text{-Vậy:} \\
$ \begin{aligned}
    T(n) &= 8^{\log_2{n}}T(\dfrac{n}{2^{\log_2{n}}}) + \log_2(n).n^3 \\
        &= n^3T(1) + n^3\log_2{n} \\
        &= n^3 + n^3\log_2{n} \\
        &= n^3(1 + \log_2{n}) \\
        &= n^3(\log_2{2} + \log_2{n}) \\
        &= n^3\log_2(2n)
\end{aligned} $\\
\section{Giải các phương trình đệ quy sau dùng phương trình đặc trưng}
\begin{enumerate}
    \item $ \begin{aligned}[t]
            T(n) &= 4T(n - 1) - 3T(n-2) \\
            T(0) &= 1 \\
            T(1) &= 2
    \end{aligned} $

    \item $ \begin{aligned}[t]
            T(n) &= 4T(n - 1) - 5T(n - 2) + 2T(n - 3) \\
            T(0) &= 0 \\
            T(1) &= 1 \\
            T(2) &= 2
    \end{aligned} $

    \item $ \begin{aligned}[t]
            T(n) &= T(n - 1) + T(n - 2) \\
            T(0) &= 0 \\
            T(1) &= 1
    \end{aligned} $

\end{enumerate}

\end{document}
